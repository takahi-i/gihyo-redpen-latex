\documentclass[a4paper, 10pt]{jarticle}

\usepackage{graphicx}
\usepackage{latexsym}
\usepackage{url}
\usepackage{ascmac,here,txfonts,txfonts}
\usepackage{color}
\usepackage{listings, jlisting}
\usepackage[justification=centering]{caption}

\title{自動文書検査ツール RedPen}
\author{伊藤 敬彦}

\begin{document}
\maketitle
\begin{abstract}
規約にしたがった文書記述をサポートするツールとして RedPen \cite{redpen} を作成した.RedPen は自然言語で
記述された入力文書の検査を一部自動化する目的で作成された.RedPen は設定を変更することで所属組織の規約
にあわせた検査が実現できる.また RedPen に足りない機能があればユーザが追加できる.

本稿では RedPen の開発の背景,提供する機能について解説する.
\end{abstract}

\section{まえがき}

ソフトウェアエンジニアや研究者は技術文書を書く機会が多く存在する.

技術文書は論文,マニュアル,ブログなど多岐にわたるが ``規約'' にしたがって記述するという共通の特徴を持つ.

文書の規約は文書の執筆者が従うべきルールである.一般に規約は集団で文書を作成する際にメンバが従うべき共通のルールとして使用される.
個人で文書を執筆する際にも,文書全体が一貫したスタイルを持つように策定される場合がある.

\bibliographystyle{plain}
\bibliography{reference,reference-j}

\end{document}
