\section{まえがき}

ソフトウェアエンジニアや研究者は技術文書を書く機会が多く存在する.

記述される技術文書は論文,マニュアル,仕様書,ブログなど多岐にわたるが,技術文書は ``規約'' にしたがって記述するという共通の特徴を持つ.

文書の規約は文書の執筆者が従うべきルールである.
一般に規約は集団で文書を作成する際にメンバが従うべき共通のルールとして使用される.
個人で文書を記述する際にも,文書全体が一貫した記述になるために策定される.

規約には一文の長さ,利用する句読点の種類(半角全角など),文書中で利用する技術単語の選択などがあり,文書を作成する組織ごとに異なる.たとえば,
アルゴリズムをアルファベットで記述する組織もあれば,カタカナに変換して記述する組織も存在する.どちらを採用しても大きな問題はないが,
規約が混在してしまうと文書の可読性が低下したり,品質に関する印象を損ねるおそれがある.

一方,C 言語や Java などの計算機プログラムも自然言語で記述された文書と同じように人によって記述され,組織毎に規約が存在する文書といえる.
プログラムの作成環境と自然言語で記述された文書の作成環境の違いの一つにプログラムの作成環境には多くの品質検査ツールが
提供され,一般的に利用されている点がある.

たとえばソフトウェアエンジニアがプログラムを作成する際には,静的プログラム解析ツールや機能テストツールを利用して品質が悪化するのを防ぐ.
静的プログラム解析ツールはコンピュータのソフトウェアの解析ツールであり,実行ファイルを実行することなく解析を行うツールである.
静的プログラム解析ツールには,コードのフォーマットが規約に従っているかを検査する(CheckStlye~\cite{checkstyle} や lint 等)ものや
潜在的なバグを発見するものなど多岐にわたる.

自然言語文書の作成でも利用できるツールは存在するが,多くが無料で利用できなかったり,研究目的で開発されたがすでに保守されていない.

RedPen はコマンドラインから利用する文書検査ツールである.ユーザは設定ファイルを変更することで所属組織にあわせた設定を構築できる.
シェルコマンドとして呼び出せるため,RedPen はバージョン管理システムなどのエンジニアが利用するツールと組み合わせて利用できる.
また日本語以外の文書にも適用でき,必要な検査を比較的簡単に追加できるという特徴をもつ.

以下の節で RedPen を作成した背景について述べ,次に RedPen の機能について紹介する.

\section{背景}
本節では RedPen を開発した背景について述べる.はじめに校正支援ツールを利用したプログラム開発について述べ,
次に自然言語文書作成環境の現状について述べる.

\subsection{ツールを活用したプログラム開発}
ソフトウェアエンジニアがプログラムを構築する際に,対象となるプログラミング言語のコンパイラ
(インタープリタ)とエディタ以外に多様なツール群を利用する.その多くはプログラムの品質を検査するツールである.

\subsubsection{プログラムの品質を検査するツール}
プログラムの品質を検査するツールにはいくつかの種類がある.以下各種類ごとに解説する.

\begin{description}
 \item[\textgt{機能テスト}] 
   機能テストツールはプログラムのクラスや関数といった部品に対して期待した振る舞いを行っているかを検査するツール群をさす.
   各言語ごとにテストツールは存在する.Java 開発には JUnit が一般的に利用されている.これらのツールを使って作成した部
   品に期待した振る舞いを定義する.テストを記述することで将来コードがまちがって変更された際に,ツールがエラーを発生してくれる.
 \item[\textgt{静的プログラム解析ツール}]
   静的プログラム解析ツールはコンパイルされる前のプログラムに対して品質や性能を検査するツール群をさす.
   
   プログラムのフォーマットの不整合を検査するツールとして lint や CheckStyle がある.これらのツールは
   一文の長さや,変数名などが設定ファイルに記述した規約に則っているかを検査し,入力プログラムが規約に
   違反した場合エラーを報告する.また,FindBugs~\cite{findbugs} は Java で記述されたプログラムの潜在的なエラーを検査する.
\item[\textgt{バージョン管理システム}]
  バージョン管理システムはプログラム開発の基盤ツールとして長く利用されている.現在では Git が有名だがそれ
  以外にも多数のバージョン管理システムが存在する.

  バージョン管理システムは登録されたプログラムの履歴を管理する.この機能から
  ソフトウェアエンジニアは過去の状態や変更内容を確認したり,過去のプログラムを復元できる.
  
  ソフトウェアエンジニアがプログラムを構築する際,まず編集したプログラムに静的プログラム解析ツールやテストツールを走らせ,
  編集した結果のプログラムに問題ないことを確認してから中央のレポジトリに変更を反映する.手元でテストや静的解析ツール群を走らせた
  後に変更を反映させることで,プログラムにバグが混入したり,品質が低減してしまうのを防ぐ.
\end{description}

\subsection{自然言語文書の開発環境}
前節で述べたようにプログラムの開発は品質の向上に寄与するツール群を駆使して行われる.一方自然言語文書の作成ではあまり利用
できるツールが整備されていない.

株式会社ジャストシステムが提供している文書校正支援ツール Just Right!~\cite{justright} は
文の誤り検査(誤字脱字,仮名遣い,慣用表現,呼応表現,ら抜き表現,同音語誤り,二重敬語),用語基準(送り仮名,漢字基準),表現
(文体の統一,重ね言葉,同一助詞の連続,二重否定)など多くの機能を提供している.
ただし Just Right! は商用製品のため無料で利用できない.

また自動で文書検査するツールに日本語表現法開発プロジェクト(PaWeL)が公開している Tomarigi~\cite{tomarigi}~\cite{tomarigi-paper} 
は無料で利用できる文書の自動検査ツールや``Chantokun''~\cite{chantokun} がある.しかしこれらのツールはコマンドラインでの利用ができない.
そのため,ソフトウェアエンジニアが Git などのレポジトリ管理ツールや他のコマンドラインツール群と組み合わせて利用するのが難しい.また,
ユーザや所属組織によって異なる規約にフィットするための規約の更新方法がシステムに提供されていない.

文法誤りの検出だけではなく訂正を行う研究に水本ら~\cite{mizumoto12english} の英文法自動誤り訂正を行ったものがあるが,手法は一般的に利用
できる形では配布されていない.
